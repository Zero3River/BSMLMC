\section{Conclusion}\label{sec:Conclusion}

In this report, we have implemented a BSML (Bilateral State-Based Modal Logic) model checker and successfully used it to validate the models, syntax, and semantics of BSML.\@ 
BSML provides a powerful framework for handling free-choice phenomena in natural language, and by using Haskell, we have built an efficient and reliable tool to evaluate more complex models and formulas.

Currently, the implementation of BSML is limited to natural deduction, and there is no way to handle assumptions in Haskell. 
Therefore, future work may involve providing a sequent calculus for the BSML system.

Additionally, BSML has many other extended versions, all of which could be implemented in Haskell in the future. For example:

\textbf{QBSML} (see \citet{Aloni2023}) extends BSML with quantification over possible worlds and states. Implementing this extension in Haskell would refine our model checker to handle richer linguistic sentences, thus enhancing the expressiveness of BSML.\@

\textbf{Bilateral Update Semantics (BiUS)} (See \citet{BiUS2023}) introduces a dynamic perspective on meaning change, incorporating updates. Implementing BiUS in the current model checker could enhance its ability to model information dynamics in discourse.

These extensions will improve the expressiveness of BSML and further demonstrate Haskell's suitability for formal semantic modeling.
