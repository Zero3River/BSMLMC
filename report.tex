% \documentclass[12pt,a4paper]{article}
% \usepackage{etex,datetime,setspace,latexsym,amssymb,amsmath,amsthm}
\usepackage{fancybox,dialogue,float,wrapfig,enumerate,microtype}
\usepackage{verbatim,xcolor,multicol,titlesec,tabularx,mdframed}

\usepackage[utf8]{inputenc}
\usepackage[pdftex]{hyperref}
\usepackage[margin=2cm,bottom=3cm,footskip=15mm]{geometry}
\parindent0cm
\parskip0.5em

\usepackage{mathtools}
\usepackage{graphicx}

% 定义一个新的“反向 models”符号
\newcommand{\leftmodels}{\mathrel{\reflectbox{$\models$}}}
% 定义 split disjunction 符号(向右旋转 90°)
\newcommand{\inqdisj}{\mathrel{\raisebox{1.5ex}{\rotatebox{-90}{$\geqslant$}}}}

\usepackage{tikz}
\usetikzlibrary{arrows,trees,positioning,shapes,patterns}
\usetikzlibrary{intersections,calc,fpu,decorations.pathreplacing}

\usepackage[T1]{fontenc} % better fonts

% Haskell code listings in our own style
\usepackage{listings,color}
\definecolor{lightgrey}{gray}{0.35}
\definecolor{darkgrey}{gray}{0.20}
\definecolor{lightestyellow}{rgb}{1,1,0.92}
\definecolor{dkgreen}{rgb}{0,.2,0}
\definecolor{dkblue}{rgb}{0,0,.2}
\definecolor{dkyellow}{cmyk}{0,0,.7,.5}
\definecolor{lightgrey}{gray}{0.4}
\definecolor{gray}{gray}{0.50}
\lstset{
  language        = Haskell,
  basicstyle      = \scriptsize\ttfamily,
  keywordstyle    = \color{dkblue},     stringstyle     = \color{red},
  identifierstyle = \color{dkgreen},    commentstyle    = \color{gray},
  showspaces      = false,              showstringspaces= false,
  rulecolor       = \color{gray},       showtabs        = false,
  tabsize         = 8,                  breaklines      = true,
  xleftmargin     = 8pt,                xrightmargin    = 8pt,
  frame           = single,             stepnumber      = 1,
  aboveskip       = 2pt plus 1pt,
  belowskip       = 8pt plus 3pt
}
\lstnewenvironment{code}[0]{}{}

% only shown, not compiled:
\lstnewenvironment{showCode}[0]{\lstset{numbers=none}}{}

% only compiled, not shown:
\newcommand{\hide}[1]{}

% will the real phi please stand up
\renewcommand{\phi}{\varphi}

% load hyperref as late as possible for compatibility
\usepackage[pdftex]{hyperref}
\hypersetup{
  pdfborder = {0 0 0},
  breaklinks = true,
  linktoc = all,
}
\pdfinfoomitdate=1
\pdftrailerid{}
\pdfsuppressptexinfo15


% \title{A Model Checker for Bilateral State-based Modal Logic (BSML)}
% \author{Me}
% \date{\today}
% \hypersetup{pdfauthor={Me}, pdftitle={A Model Checker for Bilateral State-based Modal Logic (BSML)}}

% \begin{document}

% \maketitle

% \begin{abstract}
% We give a toy example of a report in \emph{literate programming} style.
% The main advantage of this is that source code and documentation can
% be written and presented next to each other.
% We use the listings package to typeset Haskell source code nicely.
% \end{abstract}

% \vfill

% \tableofcontents

% \clearpage

% % We include one file for each section. The ones containing code should
% % be called something.lhs and also mentioned in the .cabal file.

% 
\section{How to use this?}

To generate the PDF, open \texttt{report.tex} in your favorite $\LaTeX$ editor and compile.
Alternatively, you can manually do
\texttt{pdflatex report; bibtex report; pdflatex report; pdflatex report} in a terminal.

You should have stack installed (see \url{https://haskellstack.org/}) and
open a terminal in the same folder.

\begin{itemize}
  \item To compile everything: \verb|stack build|.
  \item To open ghci and play with your code: \verb|stack ghci|
  \item To run the executable from Section\ref{sec:Main}: \verb|stack build && stack exec myprogram|
  \item To run the tests from Section\ref{sec:simpletests}: \verb|stack clean && stack test --coverage|
\end{itemize}



% \input{lib/Basics.lhs}

% \input{exec/Main.lhs}

% \input{test/simpletests.lhs}

% \section{Conclusion}\label{sec:Conclusion}

In this project, we have developed a Haskell-based model checker for BSML (Bilateral State-Based Modal Logic). 
We first provided a concise introduction to the core framework of BSML and its extension with global disjunction.
We then implemented its models, syntax, and semantics in Haskell, ensuring a formal and structured representation of the system.\@ 
To validate our implementation, we used QuickCheck to verify several key properties of BSML.\@
Additionally, we developed a web interface to make the model checker more accessible and user-friendly.
BSML offers a robust framework for capturing free-choice inferences in natural language,
and our Haskell-based implementation provides an efficient and reliable tool for analyzing complex models and formulas.


Currently, our implementation of BSML is restricted to natural deduction, and Haskell does not yet provide a way to handle assumptions within this framework.
Future work could focus on developing a sequent calculus for BSML, which would enable a more expressive proof system within the framework.
This would allow for a more comprehensive exploration of the logical properties of BSML and its applications in natural language semantics.

Additionally, BSML has many other extended versions, all of which could be implemented in Haskell in the future, such as:

\textbf{QBSML} (see \cite{Aloni2023}) extends BSML with quantification over possible worlds and states. Implementing this extension in Haskell would refine our model checker to handle richer linguistic sentences, thus enhancing the expressiveness of BSML.\@

\textbf{Bilateral Update Semantics (BiUS)} (See \cite{BiUS2023}) introduces a dynamic perspective on meaning change, incorporating updates. Implementing BiUS in the current model checker could enhance its ability to model information dynamics in discourse.

These extensions will improve the expressiveness of BSML and further demonstrate Haskell's suitability for formal semantic modeling.

In conclusion, our Haskell-based model checker for BSML provides a solid foundation for exploring the logical properties of natural language semantics.
The combination of Haskell's strong type system and functional programming paradigm with the expressive power of BSML offers a promising avenue for future research in this field.


% \addcontentsline{toc}{section}{Bibliography}
% \bibliographystyle{alpha}
% \bibliography{references.bib}

% \end{document}


\documentclass[12pt,a4paper]{article}
\usepackage{etex,datetime,setspace,latexsym,amssymb,amsmath,amsthm}
\usepackage{fancybox,dialogue,float,wrapfig,enumerate,microtype}
\usepackage{verbatim,xcolor,multicol,titlesec,tabularx,mdframed}

\usepackage[utf8]{inputenc}
\usepackage[pdftex]{hyperref}
\usepackage[margin=2cm,bottom=3cm,footskip=15mm]{geometry}
\parindent0cm
\parskip0.5em

\usepackage{mathtools}
\usepackage{graphicx}

% 定义一个新的“反向 models”符号
\newcommand{\leftmodels}{\mathrel{\reflectbox{$\models$}}}
% 定义 split disjunction 符号(向右旋转 90°)
\newcommand{\inqdisj}{\mathrel{\raisebox{1.5ex}{\rotatebox{-90}{$\geqslant$}}}}

\usepackage{tikz}
\usetikzlibrary{arrows,trees,positioning,shapes,patterns}
\usetikzlibrary{intersections,calc,fpu,decorations.pathreplacing}

\usepackage[T1]{fontenc} % better fonts

% Haskell code listings in our own style
\usepackage{listings,color}
\definecolor{lightgrey}{gray}{0.35}
\definecolor{darkgrey}{gray}{0.20}
\definecolor{lightestyellow}{rgb}{1,1,0.92}
\definecolor{dkgreen}{rgb}{0,.2,0}
\definecolor{dkblue}{rgb}{0,0,.2}
\definecolor{dkyellow}{cmyk}{0,0,.7,.5}
\definecolor{lightgrey}{gray}{0.4}
\definecolor{gray}{gray}{0.50}
\lstset{
  language        = Haskell,
  basicstyle      = \scriptsize\ttfamily,
  keywordstyle    = \color{dkblue},     stringstyle     = \color{red},
  identifierstyle = \color{dkgreen},    commentstyle    = \color{gray},
  showspaces      = false,              showstringspaces= false,
  rulecolor       = \color{gray},       showtabs        = false,
  tabsize         = 8,                  breaklines      = true,
  xleftmargin     = 8pt,                xrightmargin    = 8pt,
  frame           = single,             stepnumber      = 1,
  aboveskip       = 2pt plus 1pt,
  belowskip       = 8pt plus 3pt
}
\lstnewenvironment{code}[0]{}{}

% only shown, not compiled:
\lstnewenvironment{showCode}[0]{\lstset{numbers=none}}{}

% only compiled, not shown:
\newcommand{\hide}[1]{}

% will the real phi please stand up
\renewcommand{\phi}{\varphi}

% load hyperref as late as possible for compatibility
\usepackage[pdftex]{hyperref}
\hypersetup{
  pdfborder = {0 0 0},
  breaklinks = true,
  linktoc = all,
}
\pdfinfoomitdate=1
\pdftrailerid{}
\pdfsuppressptexinfo15


\title{A Model Checker for Bilateral State-based Modal Logic (BSML)}
\author{Group 2}
\date{\today}
\hypersetup{pdfauthor={Group 2}, pdftitle={A Model Checker for Bilateral State-based Modal Logic (BSML)}}

\begin{document}

\maketitle

\begin{abstract}

    Bilateral State-based Modal Logic (BSML), which proposed by \cite{Aloni2022}, extends classical modal logic by adopting state-based semantics and introducing a non-emptiness atom to account for free choice inferences in natural language. Despite its expressive power, no automated verification tool exists for BSML.\@ This project aims to develop a model checker for BSML, enabling automated reasoning over its logical properties.

\end{abstract}

% \vfill

\tableofcontents

\clearpage

% We include one file for each section. The ones containing code should
% be called something.lhs and also mentioned in the .cabal file.


\section{Introduction}\label{sec:Introduction}

Haskell, as a functional programming language, 
emphasizes purity (no side effects) and immutability (no data modification). 
This makes Haskell particularly well-suited for handling logical systems, 
as it provides a clean and efficient way to model and manipulate abstract structures. 

BSML is a non-classical modal logic system within formal semantics. This project implements a model checker in Haskell, 
using QuickCheck to verify the reliability of the implementation. 
Additionally, a webpage interface has been created to facilitate user interaction. 

Chapter 2 introduces the linguistic background and motivation behind BSML, along with the system's key non-classical aspects. 
Chapter 3-4 delves into the syntax and semantics of BSML, as well as the Haskell implementation, which forms the core of the model checker. 
Chapter 5 introduces QuickCheck. 
Finally, Chapter 6 explains the web implementation and provides practical usage examples.

\section{Motivition}\label{Motivition}

BSML was developed to account for \textit{Free Choice} (FC) inferences, 
where disjunctive sentences give rise to conjunctive interpretations. 
For example, the sentence ``You may go to the beach or to the cinema'' typically implies 
that ``You may go to the beach \textit{and} you may go to the cinema'' 
This inference is unexpected from a classical logical perspective, 
as disjunction does not typically imply conjunction.

The key idea in BSML is the \textit{neglect-zero tendency}, 
which posits that humans tend to disregard models that verify sentences by virtue of some empty configuration. 
BSML formalizes this tendency by introducing the \textit{nonemptiness atom} (NE), which ensures that only nonempty states are considered in the interpretation of sentences. 
This leads to the prediction of both narrow-scope and wide-scope FC inferences, as well as their cancellation under negation.

BSML has been extended in two ways:
\begin{itemize}
    \item \(\textbf{BSML}^{\inqdisj}\): This extension adds the \textit{global disjunction} \(\inqdisj\), which allows for the expression of properties that are invariant under bounded bisimulation.
    \item \(\textbf{BSML}^{\oslash}\): This extension adds the \textit{emptiness operator} \(\oslash\), which can be used to cancel out the effects of the nonemptiness atom (NE).
\end{itemize}

These extensions are expressively complete for certain classes of state properties, and natural deduction axiomatizations have been developed for each of these logics.



\input{lib/Syntax.lhs}

\input{lib/Semantics.lhs}

% \input{test/simpletests.lhs}

% \input{exec/Main.lhs}

\section{Web frontend for the model checker}\label{sec:Web}

To enhance the usability of the BSML model checker, we have developed a web-based interface using Haskell for the backend and various modern web technologies for the frontend. The web application allows users to input modal logic formulas, visualize Kripke models, and dynamically view verification results. The process works as follows:

We implemented the frontend using Next.js, KaTeX \cite{katex}, and HTML5 Canvas \cite{html5_canvas}. Users can enter models, states, and formulas through input fields, and the web application submits model-checking queries via HTTP requests to the backend.

On the backend, we use Scotty, a lightweight Haskell web framework, to handle requests from the frontend and run the model checker. Once the computation is complete, the backend returns the verification result (True/False) to the frontend.

Additionally, the frontend generates a graph representation of the Kripke model and states, providing users with a visual understanding of the verification process.

This web server makes the BSML model checker more accessible and user-friendly, allowing users to verify modal logic formulas without writing any Haskell code.

\subsection{Web-Based User Interface}
We developed a Next.js frontend that provides an intuitive user interface for building and evaluating logical models.  To handle mathematical formulas, we integrated KaTeX, a JavaScript library, which dynamically renders user-entered LaTeX formulas into HTML for clear and precise display. For visualizing Kripke models, we utilized HTML5 Canvas to dynamically draw the worlds (nodes) and relationships (edges) of the logical model. The nodes are color-coded to represent different states, and the graph updates in real-time based on user input, providing an interactive and responsive experience.

To facilitate communication with the Haskell backend, we defined a structured interface, ModelEvaluationRequest, which encapsulates the essential elements of Kripke models and logical formulas. This interface includes:

\begin{itemize}
\item universe: A list of world identifiers.
\item valuation: A mapping of worlds to the propositions that hold true in them.
\item relation: A list of relationships (edges) between worlds.
\item state: The selected states (worlds) for evaluation.
\item formula: The logical formula to be evaluated.
\item isSupport: A boolean value, true means support(|=), false means not support (=|).
\end{itemize}

The frontend sends this data as a POST request to the backend, enabling seamless evaluation and retrieval of results. 


\subsection{Formula Evaluation}
The Parser module is responsible for parsing logical formulas into the internal BSMLForm representation. It supporrs:
% \begin{itemize}
% \item Atomic propositions (e.g., p1, p2)
% \item Negation (!)
% \item Conjunction & and disjunction |
% \item Global disjunction /
% \item Diamond <>
% \end{itemize}
\begin{itemize}
    \item \textbf{Atomic propositions}: e.g., \( p_1, p_2 \)
    \item \textbf{Negation}: \( ! \) (not)
    \item \textbf{Conjunction}: \( \& \)
    \item \textbf{Disjunction}: \( | \)
    \item \textbf{Global disjunction}: \( / \)
    \item \textbf{Diamond} \(\lozenge\)
\end{itemize}

\begin{code}
pForm :: Parsec String () BSMLForm
pForm = spaces >> pCnt <* (spaces >> eof) where
  pCnt =  chainl1 pDia (spaces >> (pGdis <|> pDisj <|> pConj))
  
  pConj = char '&' >> return Con
  pDisj = char '|' >> return Dis
  pGdis = char '/' >> return Gdis

  -- Diamond operator has higher precedence than conjunction
  pDia = try pDiaOp <|> pAtom
  pDiaOp = spaces >> char '<' >> char '>' >> Dia <$> pAtom
  
  -- An atom is a variable, negation, or a parenthesized formula
  pAtom = spaces >> (pBot <|> pNE <|> pVar <|> pNeg <|> (spaces >> char '(' *> pCnt <* char ')' <* spaces))
  
  -- A variable is 'p' followed by digits
  pVar = char 'p' >> P . read <$> many1 digit <* spaces
  
  pBot = string "bot" >> return Bot
  pNE = string "ne" >> return NE

  -- A negation is '!' followed by an atom
  pNeg = char '!' >> Neg <$> pAtom

parseForm :: String -> Either ParseError BSMLForm
parseForm = parse pForm "input"

parseForm' :: String -> BSMLForm
parseForm' s = case parseForm s of
  Left e -> error $ show e
  Right f -> f
\end{code}

\subsection{Web server configuration}
The server is built using Scotty \cite{scotty} and listens for POST requests at /input:

\begin{code}
main :: IO ()
main = scotty 3001 $ do
    middleware allowCors

    post "/input" $ do
        input <- jsonData :: ActionM Input
        let modelState = inputToModelState input
            (kripkeModel, state') = modelState
            KrM universe' valuation' relation' = kripkeModel
            
            -- parser and examle formula
            result = do
              parsedFormula <- parseForm (formula input)
              return $ if isSupport input
                         then modelState |= parsedFormula  -- support
                         else modelState =| parsedFormula  -- not support
                         
            -- generate response
            finalResult = case result of
              Left err -> object [
                  "error" .= show err
                , "formula" .= formula input
                , "state" .= state'
                ]
              Right checkResult -> object [
                  "result" .= checkResult
                , "formula" .= formula input
                , "state" .= state'
                , "relation" .= show relation'
                , "relation_type" .= (if isSupport input then "support |=" else "reject =|" :: String)
                ]
            
        json finalResult
\end{code}


\subsection{Usage}
Our BSML model checker web application is accessible at https://bsmlmc.seit.me.

Here we give an example to illustrate how to use the web application.

\section{Conclusion}\label{sec:Conclusion}

In this project, we have developed a Haskell-based model checker for BSML (Bilateral State-Based Modal Logic). 
We first provided a concise introduction to the core framework of BSML and its extension with global disjunction.
We then implemented its models, syntax, and semantics in Haskell, ensuring a formal and structured representation of the system.\@ 
To validate our implementation, we used QuickCheck to verify several key properties of BSML.\@
Additionally, we developed a web interface to make the model checker more accessible and user-friendly.
BSML offers a robust framework for capturing free-choice inferences in natural language,
and our Haskell-based implementation provides an efficient and reliable tool for analyzing complex models and formulas.


Currently, our implementation of BSML is restricted to natural deduction, and Haskell does not yet provide a way to handle assumptions within this framework.
Future work could focus on developing a sequent calculus for BSML, which would enable a more expressive proof system within the framework.
This would allow for a more comprehensive exploration of the logical properties of BSML and its applications in natural language semantics.

Additionally, BSML has many other extended versions, all of which could be implemented in Haskell in the future, such as:

\textbf{QBSML} (see \cite{Aloni2023}) extends BSML with quantification over possible worlds and states. Implementing this extension in Haskell would refine our model checker to handle richer linguistic sentences, thus enhancing the expressiveness of BSML.\@

\textbf{Bilateral Update Semantics (BiUS)} (See \cite{BiUS2023}) introduces a dynamic perspective on meaning change, incorporating updates. Implementing BiUS in the current model checker could enhance its ability to model information dynamics in discourse.

These extensions will improve the expressiveness of BSML and further demonstrate Haskell's suitability for formal semantic modeling.

In conclusion, our Haskell-based model checker for BSML provides a solid foundation for exploring the logical properties of natural language semantics.
The combination of Haskell's strong type system and functional programming paradigm with the expressive power of BSML offers a promising avenue for future research in this field.


%
\section*{References}\label{sec:References}

\bibliographystyle{alpha}
\bibliography{references.bib}

\addcontentsline{toc}{section}{Bibliography}
\bibliographystyle{plainnat}
\bibliography{references.bib}

\end{document}
