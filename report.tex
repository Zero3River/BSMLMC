% \documentclass[12pt,a4paper]{article}
% \usepackage{etex,datetime,setspace,latexsym,amssymb,amsmath,amsthm}
\usepackage{fancybox,dialogue,float,wrapfig,enumerate,microtype}
\usepackage{verbatim,xcolor,multicol,titlesec,tabularx,mdframed}

\usepackage[utf8]{inputenc}
\usepackage[pdftex]{hyperref}
\usepackage[margin=2cm,bottom=3cm,footskip=15mm]{geometry}
\parindent0cm
\parskip0.5em

\usepackage{mathtools}
\usepackage{graphicx}

% 定义一个新的“反向 models”符号
\newcommand{\leftmodels}{\mathrel{\reflectbox{$\models$}}}
% 定义 split disjunction 符号(向右旋转 90°)
\newcommand{\inqdisj}{\mathrel{\raisebox{1.5ex}{\rotatebox{-90}{$\geqslant$}}}}

\usepackage{tikz}
\usetikzlibrary{arrows,trees,positioning,shapes,patterns}
\usetikzlibrary{intersections,calc,fpu,decorations.pathreplacing}

\usepackage[T1]{fontenc} % better fonts

% Haskell code listings in our own style
\usepackage{listings,color}
\definecolor{lightgrey}{gray}{0.35}
\definecolor{darkgrey}{gray}{0.20}
\definecolor{lightestyellow}{rgb}{1,1,0.92}
\definecolor{dkgreen}{rgb}{0,.2,0}
\definecolor{dkblue}{rgb}{0,0,.2}
\definecolor{dkyellow}{cmyk}{0,0,.7,.5}
\definecolor{lightgrey}{gray}{0.4}
\definecolor{gray}{gray}{0.50}
\lstset{
  language        = Haskell,
  basicstyle      = \scriptsize\ttfamily,
  keywordstyle    = \color{dkblue},     stringstyle     = \color{red},
  identifierstyle = \color{dkgreen},    commentstyle    = \color{gray},
  showspaces      = false,              showstringspaces= false,
  rulecolor       = \color{gray},       showtabs        = false,
  tabsize         = 8,                  breaklines      = true,
  xleftmargin     = 8pt,                xrightmargin    = 8pt,
  frame           = single,             stepnumber      = 1,
  aboveskip       = 2pt plus 1pt,
  belowskip       = 8pt plus 3pt
}
\lstnewenvironment{code}[0]{}{}

% only shown, not compiled:
\lstnewenvironment{showCode}[0]{\lstset{numbers=none}}{}

% only compiled, not shown:
\newcommand{\hide}[1]{}

% will the real phi please stand up
\renewcommand{\phi}{\varphi}

% load hyperref as late as possible for compatibility
\usepackage[pdftex]{hyperref}
\hypersetup{
  pdfborder = {0 0 0},
  breaklinks = true,
  linktoc = all,
}
\pdfinfoomitdate=1
\pdftrailerid{}
\pdfsuppressptexinfo15


% \title{A Model Checker for Bilateral State-based Modal Logic (BSML)}
% \author{Me}
% \date{\today}
% \hypersetup{pdfauthor={Me}, pdftitle={A Model Checker for Bilateral State-based Modal Logic (BSML)}}

% \begin{document}

% \maketitle

% \begin{abstract}
% We give a toy example of a report in \emph{literate programming} style.
% The main advantage of this is that source code and documentation can
% be written and presented next to each other.
% We use the listings package to typeset Haskell source code nicely.
% \end{abstract}

% \vfill

% \tableofcontents

% \clearpage

% % We include one file for each section. The ones containing code should
% % be called something.lhs and also mentioned in the .cabal file.

% 
\section{How to use this?}

To generate the PDF, open \texttt{report.tex} in your favorite $\LaTeX$ editor and compile.
Alternatively, you can manually do
\texttt{pdflatex report; bibtex report; pdflatex report; pdflatex report} in a terminal.

You should have stack installed (see \url{https://haskellstack.org/}) and
open a terminal in the same folder.

\begin{itemize}
  \item To compile everything: \verb|stack build|.
  \item To open ghci and play with your code: \verb|stack ghci|
  \item To run the executable from Section\ref{sec:Main}: \verb|stack build && stack exec myprogram|
  \item To run the tests from Section\ref{sec:simpletests}: \verb|stack clean && stack test --coverage|
\end{itemize}



% \input{lib/Basics.lhs}

% \input{exec/Main.lhs}

% \input{test/simpletests.lhs}

% \section{Conclusion}\label{sec:Conclusion}

In this project, we have developed a Haskell-based model checker for BSML (Bilateral State-Based Modal Logic). 
We first provided a concise introduction to the core framework of BSML and its extension with global disjunction.
We then implemented its models, syntax, and semantics in Haskell, ensuring a formal and structured representation of the system.\@ 
To validate our implementation, we used QuickCheck to verify several key properties of BSML.\@
Additionally, we developed a web interface to make the model checker more accessible and user-friendly.
BSML offers a robust framework for capturing free-choice inferences in natural language,
and our Haskell-based implementation provides an efficient and reliable tool for analyzing complex models and formulas.


Currently, our implementation of BSML is restricted to natural deduction, and Haskell does not yet provide a way to handle assumptions within this framework.
Future work could focus on developing a sequent calculus for BSML, which would enable a more expressive proof system within the framework.
This would allow for a more comprehensive exploration of the logical properties of BSML and its applications in natural language semantics.

Additionally, BSML has many other extended versions, all of which could be implemented in Haskell in the future, such as:

\textbf{QBSML} (see \cite{Aloni2023}) extends BSML with quantification over possible worlds and states. Implementing this extension in Haskell would refine our model checker to handle richer linguistic sentences, thus enhancing the expressiveness of BSML.\@

\textbf{Bilateral Update Semantics (BiUS)} (See \cite{BiUS2023}) introduces a dynamic perspective on meaning change, incorporating updates. Implementing BiUS in the current model checker could enhance its ability to model information dynamics in discourse.

These extensions will improve the expressiveness of BSML and further demonstrate Haskell's suitability for formal semantic modeling.

In conclusion, our Haskell-based model checker for BSML provides a solid foundation for exploring the logical properties of natural language semantics.
The combination of Haskell's strong type system and functional programming paradigm with the expressive power of BSML offers a promising avenue for future research in this field.


% \addcontentsline{toc}{section}{Bibliography}
% \bibliographystyle{alpha}
% \bibliography{references.bib}

% \end{document}


\documentclass[12pt,a4paper]{article}
\usepackage{etex,datetime,setspace,latexsym,amssymb,amsmath,amsthm}
\usepackage{fancybox,dialogue,float,wrapfig,enumerate,microtype}
\usepackage{verbatim,xcolor,multicol,titlesec,tabularx,mdframed}

\usepackage[utf8]{inputenc}
\usepackage[pdftex]{hyperref}
\usepackage[margin=2cm,bottom=3cm,footskip=15mm]{geometry}
\parindent0cm
\parskip0.5em

\usepackage{mathtools}
\usepackage{graphicx}

% 定义一个新的“反向 models”符号
\newcommand{\leftmodels}{\mathrel{\reflectbox{$\models$}}}
% 定义 split disjunction 符号(向右旋转 90°)
\newcommand{\inqdisj}{\mathrel{\raisebox{1.5ex}{\rotatebox{-90}{$\geqslant$}}}}

\usepackage{tikz}
\usetikzlibrary{arrows,trees,positioning,shapes,patterns}
\usetikzlibrary{intersections,calc,fpu,decorations.pathreplacing}

\usepackage[T1]{fontenc} % better fonts

% Haskell code listings in our own style
\usepackage{listings,color}
\definecolor{lightgrey}{gray}{0.35}
\definecolor{darkgrey}{gray}{0.20}
\definecolor{lightestyellow}{rgb}{1,1,0.92}
\definecolor{dkgreen}{rgb}{0,.2,0}
\definecolor{dkblue}{rgb}{0,0,.2}
\definecolor{dkyellow}{cmyk}{0,0,.7,.5}
\definecolor{lightgrey}{gray}{0.4}
\definecolor{gray}{gray}{0.50}
\lstset{
  language        = Haskell,
  basicstyle      = \scriptsize\ttfamily,
  keywordstyle    = \color{dkblue},     stringstyle     = \color{red},
  identifierstyle = \color{dkgreen},    commentstyle    = \color{gray},
  showspaces      = false,              showstringspaces= false,
  rulecolor       = \color{gray},       showtabs        = false,
  tabsize         = 8,                  breaklines      = true,
  xleftmargin     = 8pt,                xrightmargin    = 8pt,
  frame           = single,             stepnumber      = 1,
  aboveskip       = 2pt plus 1pt,
  belowskip       = 8pt plus 3pt
}
\lstnewenvironment{code}[0]{}{}

% only shown, not compiled:
\lstnewenvironment{showCode}[0]{\lstset{numbers=none}}{}

% only compiled, not shown:
\newcommand{\hide}[1]{}

% will the real phi please stand up
\renewcommand{\phi}{\varphi}

% load hyperref as late as possible for compatibility
\usepackage[pdftex]{hyperref}
\hypersetup{
  pdfborder = {0 0 0},
  breaklinks = true,
  linktoc = all,
}
\pdfinfoomitdate=1
\pdftrailerid{}
\pdfsuppressptexinfo15


\title{A Model Checker for Bilateral State-based Modal Logic (BSML)}
\author{Group 2}
\date{\today}
\hypersetup{pdfauthor={Group 2}, pdftitle={A Model Checker for Bilateral State-based Modal Logic (BSML)}}

\begin{document}

\maketitle

\begin{abstract}

    Bilateral State-based Modal Logic (BSML), which proposed by\cite{Aloni2024}, extends classical modal logic by adopting state-based semantics and introducing a non-emptiness atom to account for free choice inferences in natural language. Despite its expressive power, no automated verification tool exists for BSML.\@ This project aims to develop a model checker for BSML, enabling automated reasoning over its logical properties.

\end{abstract}

\vfill

\tableofcontents

\clearpage

% We include one file for each section. The ones containing code should
% be called something.lhs and also mentioned in the .cabal file.


\section{Introduction}\label{sec:Introduction}

Haskell, as a functional programming language, 
emphasizes purity (no side effects) and immutability (no data modification). 
This makes Haskell particularly well-suited for handling logical systems, 
as it provides a clean and efficient way to model and manipulate abstract structures. 

BSML is a non-classical modal logic system within formal semantics. This project implements a model checker in Haskell, 
using QuickCheck to verify the reliability of the implementation. 
Additionally, a webpage interface has been created to facilitate user interaction. 

Chapter 2 introduces the linguistic background and motivation behind BSML, along with the system's key non-classical aspects. 
Chapter 3-4 delves into the syntax and semantics of BSML, as well as the Haskell implementation, which forms the core of the model checker. 
Chapter 5 introduces QuickCheck. 
Finally, Chapter 6 explains the web implementation and provides practical usage examples.

\section{BSML Semantics}\label{BSML Semantics}

Bilateral State-based Modal Logic (BSML) is a modal logic that employs \textit{team semantics} (also known as state-based semantics). It was introduced to account for \textit{Free Choice} (FC) inferences in natural language, where conjunctive meanings are unexpectedly derived from disjunctive sentences. For example, the sentence "You may go to the beach or to the cinema" typically implies that "You may go to the beach \textit{and} you may go to the cinema." BSML extends classical modal logic with a \textit{nonemptiness atom} (NE), which is true in a state if and only if the state is nonempty. This extension allows BSML to formalize the \textit{neglect-zero tendency}, a cognitive tendency to disregard structures that verify sentences by virtue of some empty configuration.





The syntax of BSML is defined over a set of propositional variables \(\text{Prop}\). The formulas of BSML are generated by the following grammar:

\[
\varphi ::= p \mid \bot \mid \neg \varphi \mid (\varphi \land \varphi) \mid (\varphi \lor \varphi) \mid \Diamond \varphi \mid \text{NE}
\]

where \(p \in \text{Prop}\). The classical modal logic (ML) is the NE-free fragment of BSML.

The semantics of BSML is based on \textit{team semantics}, where formulas are interpreted with respect to sets of possible worlds (called \textit{states}) rather than single worlds. A \textit{model} \(M\) is a triple \((W, R, V)\), where:
\begin{itemize}
    \item \(W\) is a nonempty set of possible worlds,
    \item \(R \subseteq W \times W\) is an accessibility relation,
    \item \(V: \text{Prop} \to \mathcal{P}(W)\) is a valuation function.
\end{itemize}

A \textit{state} \(s\) is a subset of \(W\). The support and anti-support conditions for BSML formulas are defined recursively as follows:

\begin{align*}
M, s &\models p \quad \text{iff} \quad \forall w \in s, w \in V(p) \\
M, s &\leftmodels p \quad \text{iff} \quad \forall w \in s, w \notin V(p) \\
M, s &\models \bot \quad \text{iff} \quad s = \emptyset \\
M, s &\leftmodels \bot \quad \text{always} \\
M, s &\models \text{NE} \quad \text{iff} \quad s \neq \emptyset \\
M, s &\leftmodels  \text{NE} \quad \text{iff} \quad s = \emptyset \\
M, s &\models \neg \varphi \quad \text{iff} \quad M, s \leftmodels \varphi \\
M, s &\leftmodels  \neg \varphi \quad \text{iff} \quad M, s \models \varphi \\
M, s &\models \varphi \land \psi \quad \text{iff} \quad M, s \models \varphi \text{ and } M, s \models \psi \\
M, s &\leftmodels  \varphi \land \psi \quad \text{iff} \quad \exists t, u \subseteq s \text{ s.t. } s = t \cup u \text{ and } M, t \leftmodels \varphi \text{ and } M, u \leftmodels \psi \\
M, s &\models \varphi \lor \psi \quad \text{iff} \quad \exists t, u \subseteq s \text{ s.t. } s = t \cup u \text{ and } M, t \models \varphi \text{ and } M, u \models \psi \\
M, s &\leftmodels \varphi \lor \psi \quad \text{iff} \quad M, s \leftmodels \varphi \text{ and } M, s \leftmodels \psi \\
M, s &\models \varphi \inqdisj \psi \quad \text{iff} M, s \models \varphi \text{ or } M, s \models \psi \\
M, s &\leftmodels \varphi \inqdisj \psi \quad \text{iff} M, s \leftmodels\varphi \text{ and } M, s \leftmodels \psi\\
M, s &\models \Diamond \varphi \quad \text{iff} \quad \forall w \in s, \exists t \subseteq R[w] \text{ s.t. } t \neq \emptyset \text{ and } M, t \models \varphi \\
M, s &\leftmodels  \Diamond \varphi \quad \text{iff} \quad \forall w \in s, M, R[w] \leftmodels \varphi
\end{align*}

\input{lib/Syntax.lhs}

\input{lib/Checker.lhs}

\input{exec/Main.lhs}

\section{Conclusion}\label{sec:Conclusion}

In this project, we have developed a Haskell-based model checker for BSML (Bilateral State-Based Modal Logic). 
We first provided a concise introduction to the core framework of BSML and its extension with global disjunction.
We then implemented its models, syntax, and semantics in Haskell, ensuring a formal and structured representation of the system.\@ 
To validate our implementation, we used QuickCheck to verify several key properties of BSML.\@
Additionally, we developed a web interface to make the model checker more accessible and user-friendly.
BSML offers a robust framework for capturing free-choice inferences in natural language,
and our Haskell-based implementation provides an efficient and reliable tool for analyzing complex models and formulas.


Currently, our implementation of BSML is restricted to natural deduction, and Haskell does not yet provide a way to handle assumptions within this framework.
Future work could focus on developing a sequent calculus for BSML, which would enable a more expressive proof system within the framework.
This would allow for a more comprehensive exploration of the logical properties of BSML and its applications in natural language semantics.

Additionally, BSML has many other extended versions, all of which could be implemented in Haskell in the future, such as:

\textbf{QBSML} (see \cite{Aloni2023}) extends BSML with quantification over possible worlds and states. Implementing this extension in Haskell would refine our model checker to handle richer linguistic sentences, thus enhancing the expressiveness of BSML.\@

\textbf{Bilateral Update Semantics (BiUS)} (See \cite{BiUS2023}) introduces a dynamic perspective on meaning change, incorporating updates. Implementing BiUS in the current model checker could enhance its ability to model information dynamics in discourse.

These extensions will improve the expressiveness of BSML and further demonstrate Haskell's suitability for formal semantic modeling.

In conclusion, our Haskell-based model checker for BSML provides a solid foundation for exploring the logical properties of natural language semantics.
The combination of Haskell's strong type system and functional programming paradigm with the expressive power of BSML offers a promising avenue for future research in this field.



\section*{References}\label{sec:References}

\bibliographystyle{alpha}
\bibliography{references.bib}

\addcontentsline{toc}{section}{Bibliography}
\bibliographystyle{alpha}
\bibliography{references.bib}

\end{document}
